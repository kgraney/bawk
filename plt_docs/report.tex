\documentclass[letterpaper,11pt]{report}
\usepackage{color}
\usepackage{fullpage}
\usepackage{listings}
\usepackage{caption}
\usepackage[bookmarks=true]{hyperref}

\usepackage{fouriernc}
\usepackage[T1]{fontenc}

%\usepackage[scaled]{beramono}
%\usepackage[T1]{fontenc}

%\usepackage{inconsolata}
%\usepackage[T1]{fontenc}

\title{bawk, a binary awk}
\author{
	Kevin M.\ Graney\\
	\texttt{kmg2165@columbia.edu}
}

\lstset{ %
linewidth=\textwidth,
basicstyle=\footnotesize\renewcommand{\ttdefault}{pcr}\ttfamily,       % the size of the fonts that are used for the code
numbers=left,                   % where to put the line-numbers
numberstyle=\tiny\sf,      % the size of the fonts that are used for the line-numbers
stepnumber=1,                   % the step between two line-numbers. If it's 1 each line 
                                % will be numbered
numbersep=5pt,                  % how far the line-numbers are from the code
backgroundcolor=\color{white},  % choose the background color. You must add \RequirePackage{color}
fillcolor=\color{white},
showspaces=false,               % show spaces adding particular underscores
showstringspaces=false,         % underline spaces within strings
showtabs=false,                 % show tabs within strings adding particular underscores
frame=trlb,
tabsize=4,	                  % sets default tabsize to 2 spaces
captionpos=t,                   % sets the caption-position to top
belowcaptionskip=2pt,
breaklines=true,                % sets automatic line breaking
breakatwhitespace=false,        % sets if automatic breaks should only happen at whitespace
title={\texttt{\lstname}},                 % show the filename of files included with \lstinputlisting;
upquote=true,
numberbychapter=true,
escapeinside={\%*}{*)},         % if you want to add a comment within your code
morekeywords={*,...},           % if you want to add more keywords to the set
keywordstyle=\color[rgb]{0,0,1},
commentstyle=\color[rgb]{0.133,0.545,0.133},
stringstyle=\color[rgb]{0.627,0.126,0.941},
%keywordstyle=\bfseries,
%commentstyle=\itshape,
%stringstyle=\slshape,
frameround=fttf,
framexleftmargin=14pt,
xleftmargin=14pt,
xrightmargin=4pt,
}

\DeclareCaptionFont{white}{\color{white}}
\DeclareCaptionFormat{listing}{\setlength{\fboxsep}{2pt}\hskip-4pt\colorbox{black}{\parbox{470pt}{#1#2#3}}}
\captionsetup[lstlisting]{format=listing,labelfont=white,textfont=white, singlelinecheck=false, margin=0pt, font={bf,footnotesize}}

\begin{document}
\maketitle
\tableofcontents
\lstlistoflistings

\chapter{Language tutorial}
\chapter{Language reference manual}
\chapter{Project plan}
\chapter{Architectual design}
\chapter{Test plan}
\chapter{Lessons learned}


\chapter{Source code listing}
\lstset{language=Caml,defaultdialect=[Objective]Caml}

\section{Compiler and bytecode interpreter}
\subsection{Scanner \& parser definitions}
\subsubsection{scanner.mll}
The \texttt{scanner.mll} file contains Ocamllex specifications for the bawk scanner.
\lstinputlisting[caption=scanner.mll]{../scanner.mll}
\subsubsection{parser.mly}
The \texttt{parser.mly} file contains Ocamlyacc specifications for the bawk parser.
\lstinputlisting[caption=parser.mly]{../parser.mly}

\subsection{Interfaces}
\lstinputlisting[caption=parser\_help.mli]{../parser_help.mli}
\lstinputlisting[caption=ast\_types.mli]{../ast_types.mli}
\lstinputlisting[caption=bytecode\_types.mli]{../bytecode_types.mli}
\lstinputlisting[caption=ast.mli]{../ast.mli}
\lstinputlisting[caption=compile.mli]{../compile.mli}
\lstinputlisting[caption=bytecode.mli]{../bytecode.mli}
\lstinputlisting[caption=reader.mli]{../reader.mli}
\lstinputlisting[caption=utile.mli]{../utile.mli}

\subsection{Source Files}
\lstinputlisting[caption=bawk.ml]{../bawk.ml}
\lstinputlisting[caption=parser\_help.ml]{../parser_help.ml}
\lstinputlisting[caption=ast.ml]{../ast.ml}
\lstinputlisting[caption=compile.ml]{../compile.ml}
\lstinputlisting[caption=bytecode.ml]{../bytecode.ml}
\lstinputlisting[caption=reader.ml]{../reader.ml}
\lstinputlisting[caption=utile.ml]{../utile.ml}

\section{Test files}

\section{Misc. Files}
\lstinputlisting[caption=Makefile,language=make]{../Makefile}
\lstinputlisting[caption=run\_tests.sh,language=bash]{../run_tests.sh}


\end{document}
