The testing for this project was performed using a variety of very tiny bawk language programs designed to exercise various features of the language.  These programs are provided with the source code in the \texttt{tests} directory.  Each program is named \texttt{test-<name>.bawk} and its output is expected to be identical to the \texttt{test-<name>.bawk.out} file.  Both the source and expected output files are listed in \S\ref{sec:test-files}.

The tests are all designed to operate on the binary file \texttt{tests/lichtenstein.png}, which serves as a modestly sized sample file.  The beginning of the \texttt{lichtenstein.png} file is shown below for convenience, and the file itself can be found packaged with the bawk source.
\begin{verbatim}
0000000: 8950 4e47 0d0a 1a0a 0000 000d 4948 4452  .PNG........IHDR
0000010: 0000 0200 0000 0200 0802 0000 007b 1a43  .............{.C
0000020: ad00 0000 0970 4859 7300 000a f000 000a  .....pHYs.......
0000030: f001 42ac 3498 0000 0007 7449 4d45 07d7  ..B.4.....tIME..
0000040: 0511 0921 0919 38af 7500 0020 0049 4441  ...!..8.u.. .IDA
0000050: 5478 dab4 bd5b ac6d 5b76 1dd4 5aef 63ae  Tx...[.m[v..Z.c.
0000060: bdcf bdb7 1ed7 a9b2 2b71 d995 38b1 63f2  ........+q..8.c.
0000070: b021 4fe7 0181 8404 a280 1421 10e2 2522  .!O........!..%"
0000080: 7ef8 46fc 874f 7ef9 e40f 81f8 43e1 2bca  ~.F..O~.....C.+.
0000090: 4f82 2148 0ea0 5809 2838 ca8b d871 1c95  O.!H..X.(8...q..
00000a0: 53b6 eb75 5fe7 ec35 476f 7cf4 3ec6 1c73  S..u_..5Go|.>..s
00000b0: edb5 cfbd 65c8 aeab 5be7 eeb3 f75a 73cd  ....e...[....Zs.
00000c0: 3946 1fbd b7d6 7aeb fcbd fff6 7f2b 7a17  9F....z......+z.
00000d0: 4220 626b 6c06 2a48 9a39 e980 0114 0420  B bkl.*H.9..... 
00000e0: 22f6 d09b c075 8fab 60f4 2638 f317 ed72  "....u..`.&8...r
00000f0: d1e6 7cd8 daa5 b953 4e38 e194 2481 8289  ..|....SN8..$...
...
0059270: 3122 4a04 2150 a6c8 ff1f 00c1 c91a 2379  1"J.!P........#y
0059280: 1a27 0000 0000 4945 4e44 ae42 6082       .'....IEND.B`.
\end{verbatim}

The test programs are executed by a shell script, \texttt{run\_tests.sh}, which runs each program and compares its output to the expected values.

