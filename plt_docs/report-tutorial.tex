
The bawk language is based on pattern matching bytes and extracting values from binary format data files.

\section{A sample program}
The sample program shown in \S\ref{sec:sample-program} illustrates how to print the width and height of a given PNG image file using bawk.  This is a perfect example of an application of the bawk language: a simple binary format file for which the extraction of integer data is desired.  We will use this sample program in the next section to show how bawk is executed.

\section{Executing bawk files}
To execute a bawk file, call \texttt{bawk} with a \texttt{-e} flag and specify the file on which to operate.  The bawk code is then to be provided via standard input.
\begin{center}\tt
./bawk -e [binary filename] < [bawk filename]
\end{center}
This command will compile and run the bawk program over the given binary file.  The \texttt{-e} flag is optional and may be ommitted.  (The default operation is to execute the program if no flags are given.)

To execute the sample program, named \texttt{tests/test-util1.bawk}, we run bawk on a sample file, \texttt{tests/lichtenstein.png}, and see that the height and width of this file are both $512$ pixels.
\begin{quote}\begin{verbatim}
$ ./bawk -e tests/lichtenstein.png < tests/test-useful1.bawk
512
512
\end{verbatim}\end{quote}

To see the bytecode instructions for a given program use instead the \texttt{-c} flag.  With this flag, the bytecode instructions will be printed to standard output.
\begin{center}\tt
./bawk -c < [bawk filename]
\end{center}
If we compile our sample program for PNG files, we can see the bytecode generated, which is shown below.
\begin{quote}\begin{multicols}{2}\begin{verbatim}
$ ./bawk -c < tests/test-useful1.bawk 
        0: Ldp
        1: Ldp
        2: Beo 124
        3: Lod 0
        4: Bne 10
        5: Rdb 1
        6: Lit 137
        7: Bin Sub
        8: Bne 119
        9: Bra 14
       10: Rdb 1
       11: Lit 137
       12: Bin Sub
       13: Bne 119
       14: Rdb 1
       15: Lit 80
       16: Bin Sub
       17: Bne 119
       18: Rdb 1
       19: Lit 78
       20: Bin Sub
       21: Bne 119
       22: Rdb 1
       23: Lit 71
       24: Bin Sub
       25: Bne 119
       26: Lod 0
       27: Bne 33
       28: Rdb 1
       29: Lit 13
       30: Bin Sub
       31: Bne 119
       32: Bra 37
       33: Rdb 1
       34: Lit 13
       35: Bin Sub
       36: Bne 119
       37: Lod 0
       38: Bne 44
       39: Rdb 1
       40: Lit 10
       41: Bin Sub
       42: Bne 119
       43: Bra 48
       44: Rdb 1
       45: Lit 10
       46: Bin Sub
       47: Bne 119
       48: Lod 0
       49: Bne 55
       50: Rdb 1
       51: Lit 26
       52: Bin Sub
       53: Bne 119
       54: Bra 59
       55: Rdb 1
       56: Lit 26
       57: Bin Sub
       58: Bne 119
       59: Lod 0
       60: Bne 66
       61: Rdb 1
       62: Lit 10
       63: Bin Sub
       64: Bne 119
       65: Bra 70
       66: Rdb 1
       67: Lit 10
       68: Bin Sub
       69: Bne 119
       70: Ldp
       71: Ldp
       72: Beo 116
       73: Rdb 4
       74: Str 2
       75: Rdb 1
       76: Lit 73
       77: Bin Sub
       78: Bne 111
       79: Rdb 1
       80: Lit 72
       81: Bin Sub
       82: Bne 111
       83: Rdb 1
       84: Lit 68
       85: Bin Sub
       86: Bne 111
       87: Rdb 1
       88: Lit 82
       89: Bin Sub
       90: Bne 111
       91: Ldp
       92: Ldp
       93: Beo 108
       94: Rdb 4
       95: Str 3
       96: Rdb 4
       97: Str 4
       98: Lod 3
       99: Jsr -1
      100: Lod 4
      101: Jsr -1
      102: Bra 108
      103: Skp
      104: Beo 108
      105: Rdb 1
      106: Drp
      107: Bra 92
      108: Skp
      109: Skp
      110: Bra 116
      111: Skp
      112: Beo 116
      113: Rdb 1
      114: Drp
      115: Bra 71
      116: Skp
      117: Skp
      118: Bra 124
      119: Skp
      120: Beo 124
      121: Rdb 1
      122: Drp
      123: Bra 1
      124: Skp
      125: Skp
      126: Hlt
\end{verbatim}\end{multicols}\end{quote}

\section{Walking through the example}
The first line of the sample program in \S\ref{sec:sample-program} matches the pattern \texttt{89 "PNG" 0d0a1a0a}, where the string \texttt{"PNG"} is translated into ASCII bytes \texttt{50 4e 47}.  Bawk will scan through the input file until finding this pattern.  On the first occurrence, the block statement following will be executed, and the nested patterns will attempt to match starting at the end of the match from the higher-level pattern.  When a block exits, the pattern matching returns to executing at the position where it last began searching.

There is a special variable in bawk, \texttt{RP}, that can be used to retrieve or set the \emph{read pointer} over the binary file.  The value of \texttt{RP} represents the number of bytes from the beginning of the file.  If we modify the example program to show us \texttt{RP} at various execution points it should become clear how the variable behaves.
\begin{lstlisting}[caption=Showing how RP works]
print(RP);
/89 "PNG" 0d 0a 1a 0a/ {
	print(RP);
	/length:uint4 "IHDR"/ {
		print(RP);
		/width:uint4 height:uint4/ {
			print(RP);
		}
		print(RP);
	}
	print(RP);
}
print(RP);
RP + 100;
print(RP);
/ffee/ { print(RP); }
print(RP);
\end{lstlisting}
Running this code produces the following output on \texttt{lichtenstein.png}.
\begin{quote}\begin{verbatim}
0
8
16
24
16
8
0
0
15705
0
\end{verbatim}\end{quote}
As you can see, the \texttt{RP} pointer resets at the end of each pattern matching block, and can also be manually advanced.  Pattern matching always searches forward from \texttt{RP}.

In addition to pattern matching bawk supports dynamic variable scoping, functions, conditional statements, and mixed endianness.  See the language reference manual in Chapter \ref{chap:lrm} for details.
